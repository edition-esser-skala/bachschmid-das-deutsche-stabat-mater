\documentclass[tocstyle=ref-genre]{ees}

\shorttitle{Stabat Mater}

\begin{document}

\eesTitlePage

\eesCriticalReport{
  2  & 73   & vl 2   & grace note missing in \A1 \\
     & 109  & org    & bar in \A1: A2. \\
  3  & 101f & ob, vl & bars adjusted to bar 45f \\
  4  & 15   & vla    & 2nd \halfNote\ in \A1: f2 \\
     & 150  & vl 2   & 7th \eighthNote\ in \A1: c″8 \\
  5  & 120  & vl 1   & grace note missing in \A1 \\
  8  & 63   & cor 1  & 1st \quarterNote\ in \A1 c″4 \\
     & 63   & cor 2  & 1st \quarterNote\ in \A1 c′4 \\
     & 99   & vl 1   & 3rd \quarterNote\ in \A1: \sharp g″16–e″16–\sharp g″16–e″16 \\
     & 99   & vl 2   & 3rd \quarterNote\ in \A1: d″16–b′16–d″16–b′16 \\
     & 99   & vla    & 3rd \quarterNote\ in \A1: b′16–\sharp g′16–b′16–\sharp g′16 \\
     & 112  & vl 2   & 3rd \quarterNote\ in \A1: \sharp f″4 \\
     & 120  & A      & last \quarterNote\ in \A1: d′4 \\
     & 141  & S      & bar in \A1: b′4.–b′8–b′8–b′8 \\
  9  & 56   & A      & 1st \quarterNoteDotted\ in \A1: g′4. \\
     & 172–175 & vla 1 & in \A1 unison with vla 2 \\
  10 & 105  & vl 2   & 1st \eighthNote\ in \A1: b8 \\
}

\eesToc{
\begin{movement}{welchein}
  \voice[Tenore]
  Welch ein Anblikk?\\
  Seht die Mutter der Schmerzen,\\
  ohnmachtsvoll am Baum der Welten Erlöſung!\\
  Ach! Es blutet ihr Sohn!\\
  O! Mutterleiden!\\
  Ach! Ihr Eingebohrner!\\
  Ach! es blutet der Gottmenſch,\\
  blutet, und ſtirbt, ſo quallenvoll und ſchmächlich!\\
  Und dies muß die betrübte Mutter ſehen!\\
  O! es durchwühlet mitten ihre gebeugte Seel der Stahl der Leiden.\\
  Wie Sie ſeufzet? und mit den Schmerzen ringet?\\
  Wie Sie verlaßen daſteht?\\
  Ihrem Sohne zur Seite, Gottes Mutter, Gottes Gebenedeyte.
\end{movement}

\begin{movement}{dicherblicken}
  \voice[Tenore]
  Dich erblicken ohne Thräne,\\
  Gottes Mutter, kann ich nicht.\\
  O, ſie fließet! Aber jene\\
  kalterpreßte Augenthräne,\\
  Gottes Mutter, fließet nicht.\\
  Seh ich dich in deinem Leide,\\
  ſeh ich dich in deinem Schmerz,\\
  dich an deines Sohnes Seite,\\
  Mutter, dann weint mein Herz.
\end{movement}

\begin{movement}{jaherzens}
  \voice[Coro]
  Ja, Herzens Thränen laß uns weinen,\\
  laß ſie uns weinen nur genug\\
  am Kreuz bey jenen heiligen Gebeinen,\\
  die dorthin unſre Sinde ſchlug.\\
  Laß uns, o Mittler, klagen,\\
  ach, wir haben Wunden dir\\
  und deiner Mutter auch gegraben,\\
  wir Sindenknechte wir.
\end{movement}

\clearpage
\begin{movement}{fuerdie}
  \voice[Basso]
  Für die Laſter ſeiner Kinder,\\
  für die Müßethat der Kinder\\
  leidet Jeſus Schmach und Hohn,\\
  laßet ſich in Purpur kleiden,\\
  trägt zur Fülle ſeiner Leiden\\
  eine dorngeflochtne Kron.\\
  Seine müde Würger keichen,\\
  Streiche folgen Geißelſtreichen\\
  unter Spott und Höllenton.\\
  Duldend, gleich dem Opferlamme,\\
  naht er ſich dem Kreuzes Stamme\\
  zu des hohen Wohlthuns Lohn.\\
  Dieſe Leiden alle dulden\\
  nur zur Tilgung unſrer Schulden\\
  ſiehſt du, Mutter, deinen Sohn.
\end{movement}

\begin{movement}{verlassen}
  \voice[Tenore]
  Verlasſen!\\
  Im letzten Kampfe ganz verlasſen,\\
  o Mutter, ſiehſt du deinen Sohn,\\
  Er ſtirbt, dein Sohn, er ſtirbt.
\end{movement}

\begin{movement}{wenneinst}
  \voice[Soprano, Alto]
  Wenn einſt mein lezter Kampf beginnet,\\
  laß mich, Gott, jene Städte ſehen,\\
  wo du gehängt auf Golgatha.\\
  Wenn einſt mein Tropfen Zeit verrinnet,\\
  laß meinen Geiſt vorüber gehen,\\
  was heute deine Mutter ſah.\\
  Dein Blut iſt auch für mich geflosſen,\\
  für mich haſt du es auch vergosſen,\\
  es wird im Kampfe Kraft mir geben,\\
  der matten Seele neues Leben:\\
  Dann, Bundesmittler, ſiegen wir,\\
  und danken ewig dir.
\end{movement}

\begin{movement}{gernemutter}\enlargethispage\baselineskip
  \voice[Tenore]
  Gerne, Mutter, will ich leiden,\\
  was dein Sohn gelitten hat.\\
  Folgen will ich ihm mit Freuden\\
  auf des Lebens Dornenpfad.\\
  Will den Koſeweg verſchmähen,\\
  deinem Sohne folge ich,\\
  will mit dir am Kreuze ſtehen,\\
  Mutter, ach, ach lasſe mich.
\end{movement}

\begin{movement}{wirwollen}
  \voice[Coro]
  Wir wollen wie der Mittler leiden,\\
  das wollen wir, zu Gott gekehrt.\\
  Wir wollen ſterben, wollen es mit Freuden,\\
  weil Jeſus Tod und Sterben lehrt.\\
  Wir trozen allen Quallen, allen Schmerzen,\\
  dem Todespfeile, er iſt ſtumpf.\\
  Wir ſehn ins Grab mit unerſchrocknem Herzen,\\
  dies machet, Jeſus, dein Triumph.
\end{movement}

\begin{movement}{omutter}
  \voice[Alto]
  O Mutter aller reinen Liebe,\\
  o zürne doch mir ſchwachem nicht.\\
  Entzieh nicht, wenn ich dich betrübe,\\
  mir fallendem dein Angeſicht.\\
  Vergieb mir, deinem Kinde,\\
  das einmal ganz aus Boßheit irrt,\\
  vergieb mir, weil ſo manche Sinde\\
  aus Schwachheit nur begangen wird.
\end{movement}

\begin{movement}{undwenn}
  \voice[Basso]
  Und wenn einſt am großen Tage\\
  aller Welten Richter kommt,\\
  fürchterlich mit Schwerdt und Waage,\\
  Mutter, nihm dich meiner an.\\
  Wenn er zürnet, und die Erde\\
  unter ſeinem Fußtritt bebt,\\
  dann, o Gnadenmutter, werde\\
  meine Mittlerin.
\end{movement}

\begin{movement}{wennernte}
  \voice[Soprano]
  Wenn einſt am Erndtetage\\
  die Garben ausgereifet ſind,\\
  dann werde ich erſtehn,\\
  dann laſse unter Millionen Halmen\\
  auch meinen Halme wähn.
\end{movement}

\begin{movement}{vater}
  \voice[Coro]
  Vater, in des Sohnes Nahmen bitten wir:\\
  Schenke uns jene ſchöne Friedenshütten\\
  die er ſterbend uns erſtritten.\\
  Ach, wir wollen gerne leyden,\\
  ſchenk uns nur jene unnennbare Freuden.\\
  Amen.
\end{movement}
}

\eesScore

\end{document}
